\documentclass[uplatex]{jsreport}
\usepackage[ipaex, unicode]{pxchfon}
\usepackage{amsmath}

\newtheorem{thm}{定理}
\newtheorem{dfn}[thm]{定義}
\newtheorem{lem}[thm]{補題}
\newtheorem{prop}[thm]{命題}


\title{代数幾何学まとめノート}
\date{}

\begin{document}

\maketitle

\section{可換環}
\begin{dfn}
アーベル群 A が(単位的)可換環であるとは、
\textbf{積}と呼ばれる写像 $A \times A \to A$, $a, b \mapsto ab, \text{where}, a, b \in A$
を備えており、以下の公理を満たすときをいう: \\

任意の $a, b, c \in A$ について
\begin{enumerate}
    \item ab = ba
    \item (ab)c = a(bc)
    \item (a + b)c = ac + bc
    \item 1a = a
\end{enumerate}
ここで、$1 \in a$ を $A$ の単位元という。
\end{dfn}
以降、単に環といったら単位的可換環のことを意味すると約束する。

\begin{dfn}
    環 $A$ から $B$ への写像 $\phi: A \to B$ が環準同型写像であるとは、次の性質を満たすときをいう:
    \begin{enumerate}
        \item $\phi(a + b) = \phi(a) + \phi(b)$
        \item $\phi(ab) = \phi(a)\phi(b)$
        \item $\phi(1) = 1$
    \end{enumerate}
\end{dfn}

\end{document}

