\documentclass[uplatex]{jsreport}
\usepackage[ipaex, unicode]{pxchfon}
\usepackage{amsmath}
\usepackage{amsfonts}
\usepackage{color}

\newtheorem{thm}{定理}
\newtheorem{dfn}[thm]{定義}
\newtheorem{lem}[thm]{補題}
\newtheorem{prop}[thm]{命題}

\newcommand{\mfa}{\mathfrak{a}}
\newcommand{\mfb}{\mathfrak{b}}
\newcommand{\mfp}{\mathfrak{p}}
\newcommand{\mfq}{\mathfrak{q}}

\title{代数幾何学まとめノート}
\date{}

\begin{document}

\maketitle

\section{可換環}
\begin{dfn}
アーベル群 A が(単位的)\textbf{可換環}であるとは、
\textbf{積}と呼ばれる写像 $A \times A \to A$, $a, b \mapsto ab, \text{where}, a, b \in A$
を備えており、以下の公理を満たすときをいう: \\

任意の $a, b, c \in A$ について
\begin{enumerate}
    \item ab = ba
    \item (ab)c = a(bc)
    \item (a + b)c = ac + bc
    \item 1a = a
\end{enumerate}
ここで、$1 \in a$ を $A$ の単位元という。
\end{dfn}
以降、単に環といったら単位的可換環のことを意味すると約束する。

\begin{dfn}
    環 $A$ から $B$ への写像 $\phi: A \to B$ が\textbf{環準同型写像}であるとは、次の性質を満たすときをいう:
    \begin{enumerate}
        \item $\phi(a + b) = \phi(a) + \phi(b)$
        \item $\phi(ab) = \phi(a)\phi(b)$
        \item $\phi(1) = 1$
    \end{enumerate}
\end{dfn}

\begin{dfn}
    環 $A$ の部分集合 $\mfa$ が\textbf{イデアル}であるとは、$\mfa$ が次の性質を満たすときをいう。
    \begin{enumerate}
        \item $\mfa$は加法に関して部分群である。すなわち
        \begin{enumerate}
            \item $0 \in \mfa$
            \item $\forall a \in \mfa, -a \in \mfa$
            \item $\forall a, \forall b \in \mfa$, $a + b \in \mfa$
        \end{enumerate}
        \item $\forall a \in \mfa, \forall x \in A$, $ax \in A$.
    \end{enumerate}
\end{dfn}

\begin{dfn}
    環 $A$ のイデアル $\mfp$ が\textbf{素イデアル}であるとは、$\mfp$ が次の性質を満たすときをいう。
    \[
        p, q \in A \text{について、\ } pq \in \mfp \text{ならば} p \in \mfp \text{または} q \in \mfp.
    \]
\end{dfn}

\begin{dfn}
    (Wikipedia) 拡大体の{\bf 超越次数}とは、体の拡大$L/K$の大きさのある種のかなり粗いはかり方である。
    きちんと言えば、$K$上代数的に独立な$L$の部分集合の最も大きい濃度として定義される。
\end{dfn}

\begin{dfn}
体 $k$ の超越次数1の有限生成拡大体$K$を{\bf 1次元関数体と呼ぶ}。

\textcolor{red}{ここでは$K$として$k$上の1変数有理関数体$k(X)$を思い浮かべておけばよいはず。}
\end{dfn}

$K$ を $k$ 上の1次元関数体とする。
$C_K$ を $K/k$ の DVR すべてのなす集合とする。$C_K$の元を点とも呼ぶ。
\begin{center}
    $C_K \ni P \leftrightarrow R_P$ ($P$ に対応するDVR).
\end{center}

\begin{dfn}
    {\bf 抽象非特異曲線}とは $K$ を $k$ 上の1次元関数体として、開部分集合 $U \subseteq C_K$ である。
    ただし $U$ には誘導位相を与え、開部分集合上の正則関数の概念を $C_K$ の場合から定める。
\end{dfn}


\end{document}

